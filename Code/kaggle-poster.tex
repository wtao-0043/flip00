%%
%% This is file `tikzposter-template.tex',
%% generated with the docstrip utility.
%%
%% The original source files were:
%%
%% tikzposter.dtx  (with options: `tikzposter-template.tex')
%%
%% This is a generated file.
%%
%% Copyright (C) 2014 by Pascal Richter, Elena Botoeva, Richard Barnard, and Dirk Surmann
%%
%% This file may be distributed and/or modified under the
%% conditions of the LaTeX Project Public License, either
%% version 2.0 of this license or (at your option) any later
%% version. The latest version of this license is in:
%%
%% http://www.latex-project.org/lppl.txt
%%
%% and version 2.0 or later is part of all distributions of
%% LaTeX version 2013/12/01 or later.
%%


\documentclass{tikzposter} %Options for format can be included here

\usepackage{todonotes}

\usepackage[tikz]{bclogo}
\usepackage{lipsum}
\usepackage{amsmath}

\usepackage{booktabs}
\usepackage{longtable}
\usepackage[absolute]{textpos}
\usepackage[it]{subfigure}
\usepackage{graphicx}
\usepackage{cmbright}
%\usepackage[default]{cantarell}
%\usepackage{avant}
%\usepackage[math]{iwona}
\usepackage[math]{kurier}
\usepackage[T1]{fontenc}


%% add your packages here
%\usepackage{hyperref}
% for random text
\usepackage{lipsum}
\usepackage[english]{babel}
\usepackage[pangram]{blindtext}

\colorlet{backgroundcolor}{blue!10}

% Title, Author, Institute
\title{HOUSE PRICES-ADVANCED REGRESSION TECHNIQUES}
\author{Tao Wang}
\institute{University of Chinese Academy of Sciences, China \\
}
%\titlegraphic{logos/tulip-logo.eps}

%Choose Layout
\usetheme{Wave}

%\definebackgroundstyle{samplebackgroundstyle}{
	%\draw[inner sep=0pt, line width=0pt, color=red, fill=backgroundcolor!30!black]
	%(bottomleft) rectangle (topright);
	%}
%
%\colorlet{backgroundcolor}{blue!10}
\begin{document}
	\colorlet{blocktitlebgcolor}{blue!23}
	
	% Title block with title, author, logo, etc.
	\maketitle
	\begin{columns}
	\column{0.5}
	\block{Introduction}{
		For home buyers, they generally do not buy homes with basements or near
		railroads, and there are other features of a home that can even much more influence
		the price of a home than the number of bedrooms. 
		\begin{itemize}
			\item Project Evaluation 
			There are 1459 data in the test set, and the output contains ID numbers and
			predicted house prices. Submissions are evaluated on Root-Mean-Squared-Error
			(RMSE) between the logarithm of the predicted value and the logarithm of the
			observed sales price.
			\item File Description
			\begin{itemize}
				\item train.csv - the training set
				\item test.csv - the test set
				\item data˙description.txt - full description of each column, originally prepared by Dean De Cock but lightly edited to match the column names used here
				\item sample˙submission.csv - a benchmark submission from a linear regression on year and month of sale, lot square footage, and number of bedrooms
			\end{itemize}
			
		\end{itemize}
	}
	\block{Data Analysis}{
		There are 79 variables of inconsistent types, some discrete and some continuous, and after data processing, the variables of type object can be seen as follows:\\
		\begin{center}
			\begin{tabular}{c c c c}
				\toprule
				%\centering
				\# & \texttt{Column}  & \texttt{Non-Null Count} & \texttt{Dtype} \\
				\midrule
				$0$
				&  {$MSZoning$} &  {$1460 non-null$} &  {$object$}  \\
				$1$
				&  {$Street$} &  {$1460 non-null$}&  {$object$} \\
				$2$
				&  {$Alley$} &  {$91 non-null$} &  {$object$} \\
				$3$
				&  {$LotShape$} &  {$1460 non-null$}&  {$object$} \\
				$...$
				&  {$...$} &  {$...$} &  {$object$}\\
				\bottomrule
			\end{tabular}
		\end{center}
		The characteristics of variables of type object are represented by string,such as:
		\begin{center}
			\begin{tabular}{c c}
				\toprule
				%\centering
				\multicolumn{2}{c}{Mszoning:Identifies the general zoning classfication of the sale} \\ 
				\midrule
				$A$ &  {$Agriculture$}  \\
				$C$ &  {$Commercial$}  \\
				$...$ &  {$...$}  \\
				\bottomrule
			\end{tabular}
		\end{center}
		The numeric variables are as follows:
		\begin{center}
			\begin{tabular}{c c c c}
				\toprule
				%\centering
				\# & \texttt{Column}  & \texttt{Non-Null Count} & \texttt{Dtype} \\
				\midrule
				$0$
				&  {$MSZoning$} &  {$1460 non-null$} &  {$int64$}  \\
				$1$
				&  {$Street$} &  {$1460 non-null$}&  {$int64$} \\
				$2$
				&  {$Alley$} &  {$1201 non-null$} &  {$float64$} \\
				$3$
				&  {$LotShape$} &  {$1460 non-null$}&  {$int64$} \\
				$...$
				&  {$...$} &  {$...$} &  {$...$}\\
				\bottomrule
			\end{tabular}
		\end{center}
	}
	\block{Conclusion}{
	The difference between the house price prediction experiment and the usual data
	classification is that the data classification only needs to give the data a specific
	prediction result and then analyze the correct rate. This time, however, the house
	price prediction is to give the prediction result to a more detailed point, and then
	to analyze the error. I think this kind of prediction is more difficult to go up to a
	better result, and the number of data sets itself is not large. I think the training
	effect can be improved by increasing the dataset through data expansion.
	In conclusion, I learned a lot about the new training model in this experiment.
	For the xgboost model, more adjustments are needed to increase the number of
	parameters and the depth of the tree to prevent underfitting.
	}
	\column{0.5}
	\block{Data Pre-processing}{
		First we count the missing data in the training and test sets, and the statistics are as follows:\\
		\begin{itemize}
			\item Number of missing data in the training set:
			\begin{center}
				\begin{tabular}{c c c }
					\toprule
					%\centering
					$type$ & \texttt{num}  & \texttt{persent} \\
					\midrule
					$PoolQC$ &  {$1453$} &  {$0.995$}   \\
					$MiscFecture$ &  {$1406$} &  {$0.963$}   \\
					$Alley$ &  {$1369$} &  {$0.938$}   \\
					$Fence$ &  {$1179$} &  {$0.808$}   \\
					$FireplaceQu$ &  {$690$} &  {$0.472$}   \\
					$LotFronttage$ &  {$259$} &  {$0.177$}   \\
					$GarageType$ &  {$81$} &  {$0.055$}   \\
					$GarageYrBlt$ &  {$81$} &  {$0.555$}   \\
					$GarageFinish$ &  {$81$} &  {$0.555$}   \\
					{$...$} &  {$...$}  &  {$...$}\\
					\bottomrule
				\end{tabular}
			\end{center}
		\end{itemize}
		\begin{itemize}
			\item Number of missing data in the test set:
			\begin{center}
				\begin{tabular}{c c c }
					\toprule
					%\centering
					$type$ & \texttt{num}  & \texttt{persent} \\
					\midrule
					$PoolQC$ &  {$1456$} &  {$0.997$}   \\
					$MiscFecture$ &  {$1408$} &  {$0.964$}   \\
					$Alley$ &  {$1352$} &  {$0.926$}   \\
					$Fence$ &  {$1169$} &  {$0.800$}   \\
					$FireplaceQu$ &  {$730$} &  {$0.5$}   \\
					$LotFronttage$ &  {$227$} &  {$0.155$}   \\
					$GarageYrBlt$ &  {$78$} &  {$0.053$}   \\
					$GarageFinish$ &  {$78$} &  {$0.053$}   \\
					{$...$} &  {$...$}  &  {$...$}\\
					\bottomrule
				\end{tabular}
			\end{center}
			\item Remove the feature PoolQC, MiscFeature, Alley,Fence and FireplaceQu
			\item Use the average to fill in 8 missing numbers of masvnrarea, and use the mode to fill in the missing data of LotFrontage and GarageYrlBlt.
			\item Use log transformation to reduce skewness.
			\item Delete unreasonable data
			\item Use mode to fill in the missing data of string type.
		\end{itemize}
	}
	\block{Training Result}{
		\begin{itemize}
			\item Training Result Summary
		\end{itemize}
		\begin{center}
			\begin{tabular}{c c }
				\toprule
				%\centering
				$method$ & \texttt{result}  \\
				\midrule
				$Linear Regression (low_corr deleted)$ &  {$0.7313$}   \\
				$Random Forest(low_corr deleted)$ &  {$0.16099$}   \\
				$K-neighbor(low_corr deleted)$ &  {$0.22111$}   \\
				$Adaboost Regressor(low_corr deleted)$ &  {$0.22124$}    \\
				$Xgboost(low_corr deleted)$ &  {$0.0.60098$}    \\
				$Xgboost$ &  {$0.17429$}    \\
				$Gradient Boosting Regressor(low_corr deleted)$ &  {$0.15922$}    \\
				$Gradient Boosting Regressor$ &  {$0.14173$}     \\
				\bottomrule
			\end{tabular}
		\end{center}
	}
	\end{columns}
	\colorlet{notebgcolor}{blue!20}
	\colorlet{notefrcolor}{blue!20}
	\note[targetoffsetx=-13cm, targetoffsety=-24cm,rotate=0,angle=180,radius=8cm,width=.96\textwidth,innersep=.4cm]
	{
		\begin{minipage}{0.3\linewidth}
			\centering
			\includegraphics[width=24cm]{./graphics/logos/tulip-wordmark.eps}
		\end{minipage}
	}
\end{document}